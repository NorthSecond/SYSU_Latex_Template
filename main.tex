%!TeX program = xelatex
\documentclass{SYSUReport}

% 根据个人情况修改
\headl{左侧页眉}
\headc{中间页眉}
\headr{右侧页眉}
\lessonTitle{XXX课程报告}
\reportTitle{XXX研究}
\stuname{张三}
\stuid{20300000}
\inst{软件工程学院}
\major{软件工程}
\date{\today}

\begin{document}

% =============================================
% Part 1: 封面
% =============================================
\cover

% =============================================
% Part 2: 摘要
% =============================================
%\begin{abstract}
%
%在此填写摘要内容
%
%\end{abstract}
% \newpage

\thispagestyle{empty} % 首页不显示页码

% =============================================
% Part 3: 目录页
% =============================================
\newpage
% 重置页码,并使用罗马数字
\pagenumbering{Roman}
\setcounter{page}{1}
\tableofcontents

% =============================================
% Part 4: 正文内容
% =============================================
\newpage
% 重置页码,并使用阿拉伯数字
\pagenumbering{arabic}
\setcounter{page}{1}

%%可选择这里也放一个标题
%\begin{center}
%    \title{ \Huge \textbf{{标题}}}
%\end{center}

\section{模板说明}
本模板主要适用于一些课程的平时论文以及期末论文,默认页边距为2.5cm,中文宋体,英文Times New Roman,字号为12pt(小四)。

编译方式:\verb|xelatex -> bibtex -> xelatex*2|


默认模板文件由以下四部分组成:
\begin{itemize}
    \item \texttt{main.tex} 主文件
    \item \texttt{reference.bib} 参考文献,使用bibtex
    \item \texttt{SYSUReport.sty} 文档格式控制,包括一些基础的设置,如页眉、标题、姓名等
    \item \texttt{figures} 放置图片的文件夹
\end{itemize}

模板不需要前往\texttt{SYSUReport.sty} 对标题、姓名、学号、院所、页眉等进行设置,只需要在\texttt{main.tex} 中设置好对应的变量参数就好了,当然也可以去\texttt{SYSUReport.sty}中修改对应的设置,制作属于你自己的模板。

默认带有封面页以及目录页,页码从目录页开始。

\section{一些插入功能}
\subsection{插入公式}
行内公式$v-\varepsilon+\phi=2$。

插入行间公式如\autoref{Euler}:
\begin{equation}
    v-\varepsilon+\phi=2
    \label{Euler}
\end{equation}

\subsection{插入图片}
SYSU校徽如\autoref{fig:SYSU}所示,注意这里使用了\verb|~\autoref{}|命令,也就是会自动生成“图”“式”等前缀,无需手动输入。

\begin{figure}[!htbp]
    \centering
    \includegraphics[width =.4\textwidth]{sysu.jpg}
    \caption{中山大学}
    \label{fig:SYSU}
\end{figure}

插入上面图片的代码:

\begin{verbatim}
    \begin{figure}[!htbp]
        \centering
        \includegraphics[width =.4\textwidth]{sysu.jpg}
        \caption{中山大学}
        \label{fig:SYSU}
    \end{figure}
\end{verbatim}

\subsection{插入文本框}
本模板定义了一个圆角灰底的文本框,使用简化命令\verb|\tbox{}|即可,如果你不喜欢,可以前往 \texttt{SYSUReport.sty}对其进行修改。

\tbox{
    这是一个圆角灰底的文本框
}

\subsection{插入表格}
本模板文件如\autoref{doc}所示。
\begin{table}[!htbp]
    \centering
    \begin{tabular}{l  | l}
        \hline
        文件名                  & 说明         \\
        \hline
        \texttt{main.tex}       & 主文件       \\
        \texttt{reference.bib}  & 参考文献     \\
        \texttt{UCASReport.sty} & 文档格式控制 \\
        \texttt{figures}        & 图片文件夹   \\
        \hline
    \end{tabular}
    \caption{本模板文件组成}
    \label{doc}
\end{table}

\section{定理环境}
\begin{Theorem}
\end{Theorem}

\begin{Lemma}
\end{Lemma}

\begin{Corollary}
\end{Corollary}

\begin{Proposition}
\end{Proposition}

\begin{Definition}
\end{Definition}

\begin{Example}
\end{Example}

\begin{proof}
\end{proof}

\subsection{插入参考文献}
直接使用\verb|\cite{}|即可。

例如:


\textit{ 此处引用了文献\cite{0Isaac}。此处引用了文献\cite{2016The}}


引用过的文献会自动出现在参考文献中。

\section{写在最后}

\subsection{致谢}

本项目在\href{https://github.com/jweihe/UCAS_Latex_Template}{中国科学院大学模板}的基础上进行修改,感谢原作者的辛勤付出。

\subsection{发布地址}
\begin{itemize}
    \item Github: \url{}
    \item Overleaf:  \url{}
\end{itemize}

% =============================================
% Part 5: 参考文献
% =============================================
%在reference.bib文件中填写参考文献,此处自动生成

\newpage
\bibliographystyle{ieeetr}
\bibliography{reference.bib}

\end{document}